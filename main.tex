\documentclass [14pt,xcolor=dvipsnames,aspectratio=169]{beamer} 

\usepackage[czech]{babel}
\usepackage{svg}
\usepackage[natbib=true,style=authoryear,backend=bibtex,useprefix=true]{biblatex}
% \documentclass [14pt,xcolor=dvipsnames,aspectratio=169]{beamer} 
\usetheme{metropolis}
\setbeamertemplate{caption}{\raggedright\insertcaption\par}
\metroset{block=fill}

\definecolor{mDarkBrown}{RGB}{45, 8, 36}
\definecolor{mDarkTeal}{RGB}{45, 8, 36}
\definecolor{mLightBrown}{RGB}{229, 0, 168}
\definecolor{mLightGreen}{RGB}{229, 0, 168}

\addbibresource{references.bib}
\newcommand{\filip}[1]{\textcolor{teal} {F: #1}}
\newcommand{\julia}[1]{\textcolor{red} {J: #1}}
\newcommand{\amanda}[1]{\textcolor{blue} {A: #1}}
\newcommand{\rhett}[1]{\textcolor{green} {R: #1}}

\def\iff{\leftrightarrow}
\usepackage{csquotes}
\usepackage[utf8]{inputenc}
\usepackage{tikz}
\usetikzlibrary{positioning,calc}
\usetikzlibrary{babel}
\DeclareMathOperator*{\E}{\mathbb{E}}
\usepackage{dsfont}

\newtheorem{defi}[equation]{Definice}
\newtheorem{thm}[equation]{Věta}

\def\problem{\textsc}
\def\netlearn{\problem{Network Learning}}
\def\bnetlearn{\problem{Bayesian Network Learning}}
\def\mdnetlearn{\problem{Majority Dynamics Network Learning}}
\def\network{\mathcal{N}}
\def\absolute#1{\left\lvert #1 \right\rvert }
\def\deq{\mathbin{:=}}
\DeclareMathOperator{\Berd}{Ber}
\def\berd{\fce{\Berd}}
\def\fce#1#2{#1\!\left(#2\right)}
\def\fceb#1#2{#1\!\left[ #2 \right]}
\def\suchthat{\;\vert\;}
\def\pr#1{\fceb{\Pr}{#1}}
\def\prt#1{\fceb{\Pr}{\text{#1}}}
\def\lr#1#2#3{\fce{\mathcal{L}}{#1, #2, #3}}
\def\clr#1#2#3{\fce{\mathcal{L}}{#1, #2, #3}}
\def\olr{\fce{\mathcal{L}^*}}
\def\oclr{\fce{\mathcal{L}^*}}
\def\cell#1{\fce{\mathbb{C}}{#1}}
\def\gadget#1{\fce{\mathbb{G}}{#1}}
\def\vars{\chi}

\setbeamertemplate{caption}{\raggedright\insertcaption\par}

\title{Maximizing Truth Learning in a Social Network is \np{}-hard}
\author[1]{Filip \'{U}radn\'{i}k\textsuperscript{1,\textasteriskcentered} \and Amanda Wang\textsuperscript{2,\textasteriskcentered} \and Jie Gao\textsuperscript{3}}

\institute{
    \textsuperscript{1}Univerzita Karlova, Praha, ČR\hfill \\
    \textsuperscript{2}Princeton University, Princeton, New Jersey, USA\hfill \\
\textsuperscript{3}Rutgers University, Piscataway, New Jersey, USA \\
\textsuperscript{\textasteriskcentered}Společné první autorství.

\vspace{1em}
{\tiny\color{gray}
\texttt{uradnik@kam.mff.cuni.cz}
\hspace{1em}\textbullet\hspace{1em}
\texttt{furadnik.github.io}}
}
\date{25. září 2025}

\def\assignment{\mathcal{A}}
\def\problemClass{\textsf}
\def\np{\problemClass{NP}}

\begin{document}

\maketitle

\begin{frame}{Základní model}
    Ground truth $ \theta \sim \berd{q} $
    \begin{figure}
\begin{tikzpicture}[
	stff/.style={circle, draw=black, thick},
	stffc/.style={rectangle, draw=black, thick, minimum size=30},
 scale=1,transform shape
	]
	  \node[stff]	(1)                 {1};
		\node[stff]	(2)    [right=of 1] {2};
		\node[stff]	(3)    [right=of 2] {3};
		\node[stff]	(4)    [right=of 3] {4};
		\node[stff]	(5)    [right=of 4] {5};
		\node	(dots)    [right=of 5] {\ldots};

		\onslide<2->{
			\node(s1)		[above=of 1] {$ s_1 $};
			\node(s2)		[above=of 2] {$ s_2 $};
			\alert<5>{\node(s3)		[above=of 3] {$ s_3 $};}
			\node(s4)		[above=of 4] {$ s_4 $};
			\node(s5)		[above=of 5] {$ s_5 $};
		}

		\onslide<3->{
			\alert<5>{\node(a1)		[above=of s1] {$ a_1 $};}
		}
		\onslide<4->{
			\alert<5>{\node(a2)		[above=of s2] {$ a_2 $};}
		}
		% \onslide<5->{
		% 	\node(a3)[above=of s3] {$ a_3 $};
		% 	\node(a4)[above=of s4] {$ a_4 $};
		% 	\node(a5)[above=of s5] {$ a_5 $};
		% }
	\end{tikzpicture}
\end{figure}

\end{frame}

\begin{frame}{Rozhodovací pravidla}
    \begin{itemize}
        \item<1-> Bayesovské pravidlo -- nejpravděpodobněji správná akce,
        \item<2-> \textquote{Majority Vote} -- vyberu akci která se mezi předchůdci a mým signálem nachází nejčastěji.
    \end{itemize}
\end{frame}

\begin{frame}{Kaskádování}
    Ground truth $ \theta = 1 $
    \begin{figure}
\begin{tikzpicture}[
	stff/.style={circle, draw=black, thick},
	stffc/.style={rectangle, draw=black, thick, minimum size=30},
 scale=1,transform shape
	]
	  \node[stff]	(1)                 {1};
		\node[stff]	(2)    [right=of 1] {2};
		\node[stff]	(3)    [right=of 2] {3};
		\node[stff]	(4)    [right=of 3] {4};
		\node[stff]	(5)    [right=of 4] {5};
		\node	(dots)    [right=of 5] {\ldots};

		\onslide<2->{
			\node(s1)		[above=of 1] {$ 0 $};
			\node(s2)		[above=of 2] {$ 0 $};
		}
		\onslide<4->{
			\node(s3)[above=of 3] {$ 1 $};
			\node(s4)[above=of 4] {$ 1 $};
			\node(s5)[above=of 5] {$ 1 $};
		}

		\onslide<3->{
			\node(a1)		[above=of s1] {$ 0 $};
			\node(a2)		[above=of s2] {$ 0 $};
		}
		\onslide<5->{
			\node(a3)[above=of s3] {$ 0 $};
		}
		\onslide<6->{
			\node(a4)[above=of s4] {$ 0 $};
			\node(a5)[above=of s5] {$ 0 $};
		}
	\end{tikzpicture}
\end{figure}

\end{frame}

\begin{frame}{Social network}
\begin{defi}[Social network]
   \emph{Social network} je $ \network \deq \left( G,q,p \right) $, kde \begin{enumerate}
       \item $ G = \left( V,E \right) $ je (orientovaný) graf,
       \item $ q \in \left( 0,1 \right) $ pravděpodobnost že $\theta = 1$,
       \item $ p \in \left( \frac 12, 1 \right) $ je pravděpodobnost, že signál $s_v \in \{0,1\}$ je správně.
   \end{enumerate}
\end{defi}
\end{frame}

\begin{frame}{Rozšířený model}
    Agenti ohlašují akce v \emph{pořadí} $\sigma \in \Sigma_n$.
    
    Agent $v \in V$ má přístup k \begin{itemize}
        \item jeho soukromé informaci $s_v$,
        \item akcím jeho sousedů, kteří už akce ohlašovali \[
        N_v = \{u \in V \suchthat uv \in E \land \sigma(v) > \sigma(u)\}.
        \]
    \end{itemize}
\end{frame}

\begin{frame}{Rozhodovací pravidla}
    \begin{itemize}
        \item<1-> Bayesovská inference: \[
    \mu^B(s_v, N_v) = \begin{cases}
        1 & \text{$\pr{\theta = 1 \suchthat s_v, N_v} > \frac 12$}, \\
        0 & \text{$\pr{\theta = 0 \suchthat s_v, N_v} > \frac 12$}, \\
        \berd{\frac 12} & \text{jinak.}
    \end{cases}
        \]
        \item<2-> Majority Vote \[
    \mu^M(s_v, N_v) = \begin{cases}
        1 & \text{$s_v + \sum_{u \in N_v} a_u > \frac 12 (|N_v| + 1)$}, \\
        0 & \text{$s_v + \sum_{u \in N_v} a_u < \frac 12 (|N_v| + 1)$}, \\
				s_v & \text{jinak.}
    \end{cases}
        \]
    \end{itemize}
\end{frame}

\begin{frame}{Learning rate}
\begin{defi}
    Jako \emph{learning rate} (míru učení) $ \network $ při pořadí $ \sigma $ a rozhodovacím pravidlu $\mu$ označíme \[
		\clr \network \sigma \mu \deq  \frac 1n\sum_{v \in V} \Pr_{\theta, s}[a_v = \theta].
	\]
\end{defi}
\end{frame}

\begin{frame}{Rozhodovací problém}
\begin{defi}[{\netlearn{}}]
        Nechť rozhodovací pravidlo $\mu$ je dané.
        \netlearn{} je problém rozhodnout pro $ \network $ a $ \varepsilon \in (0,1)$, jestli \[
            \max_{\sigma \in \Sigma_n} \lr \network \sigma \mu \geq 1-\varepsilon.
        \]
\end{defi}
\end{frame}

\begin{frame}{\np{}-hardness}
    \begin{thm}
        \netlearn{} is \np{}-těžký pro $ \mu = \mu^B $ i $ \mu = \mu^M $.
    \end{thm}
    \begin{itemize}
        \item<2-> Redukce z 3-SATu.
        \item<3-> \emph{Cíl:} Pro danou formuli $\varphi$ vybudujeme $\network$ a zvolíme $\varepsilon$, tž. \[
           \mathcal{L} \geq 1-\varepsilon \quad\iff\quad \text{$ \varphi $ je splnitelná.}
        \] 
    \end{itemize}
\end{frame}

\begin{frame}{Idea důkazu}
    \begin{align*}
        \text{Pořadí $ \sigma \in \Sigma_n $}\qquad \mapsto \qquad \text{Přiřazení k proměnným}
    \end{align*}
    
\end{frame}

\begin{frame}{Buňka pro proměnnou}
\begin{columns}
 \column{0.6\textwidth}
 \onslide<2->{ $x=T \quad \leftrightarrow\quad \sigma(x) > \sigma(\lnot x)$}
\column{0.4\textwidth}
    \begin{figure}
        \centering
\begin{tikzpicture}[
	stff/.style={circle, draw=black, very thick, minimum size=30},
	]
		%Nodes
		\node[stff]        (x)                  {$ x $};
		\node[stff]        (nx)   [below=of x]   {$ \lnot x $};
		\node[stff]        (d)   [left=of x,yshift=-30]   {$ d_x $};

		%Lines
		\draw[->, very thick] (d)  to  (x);
		\draw[->, very thick] (d)  to (nx);
		\draw[<->, very thick] (x)  to (nx);
	\end{tikzpicture}
        \caption{\emph{Buňka} proměnné $x$.
        }
    \end{figure}

\end{columns}
\end{frame}

\begin{frame}{Celá konstrukce}
    \begin{figure}
\begin{tikzpicture}[
	stff/.style={circle, draw=black, thick, minimum size=30},
	stffc/.style={rectangle, draw=black, thick, minimum size=30},
 scale=0.65,transform shape
	]
		%Nodes
		\node[stff]        (x_i)                  {$ x $};
		\node[stff]        (nx_i)   [below=of x_i,yshift=10]   {$ \lnot x $};
		\node[stff]        (d_i)   [left=of x_i,yshift=-25,xshift=5]   {$ d_x $};

		\node[stff]        (x_j)    [below=of nx_i,yshift=10]  {$ y $};
		\node[stff]        (nx_j)   [below=of x_j,yshift=10]   {$ \lnot y $};
		\node[stff]        (d_j)   [left=of x_j,yshift=-25,xshift=5]   {$ d_y $};

		\node[stff]        (x_k)    [below=of nx_j,yshift=10]  {$ z $};
		\node[stff]        (nx_k)   [below=of x_k,yshift=10]   {$ \lnot z $};
		\node[stff]        (d_k)   [left=of x_k,yshift=-25,xshift=5]   {$ d_z $};

		%Lines
		\draw[->, thick] (d_i)  to  (x_i);
		\draw[->, thick] (d_i)  to (nx_i);
		\draw[<->, thick] (x_i)  to (nx_i);

		\draw[->, thick] (d_j)  to  (x_j);
		\draw[->, thick] (d_j)  to (nx_j);
		\draw[<->, thick] (x_j)  to (nx_j);

		\draw[->, thick] (d_k)  to  (x_k);
		\draw[->, thick] (d_k)  to (nx_k);
		\draw[<->, thick] (x_k)  to (nx_k);

		\onslide<2->{
			\node[stffc]	(C_1)    [right=of x_i,yshift=-5,xshift=200]  {$ C_1 $};
			\node[stffc]	(C_2)    [below=of C_1]  {$ C_2 $};
			\node[stffc]	(C_3)    [below=of C_2]  {$ C_3 $};
			\node[stffc]	(C_4)    [below=of C_3]  {$ C_4 $};
			\node[stffc]	(C_5)    [below=of C_4]  {$ C_5 $};
		}
		
		\onslide<3->{
			\alert<4>{
				\draw[->, thick] (x_i)   to (C_1);
				\draw[->, thick] (x_j)   to (C_1);
				\draw[->, thick] (nx_k)  to (C_1);
			}

			\draw[->, thick] (nx_i)   to (C_2);
			\draw[->, thick] (x_j)   to (C_2);
			\draw[->, thick] (nx_k)  to (C_2);

			\draw[->, thick] (nx_i)   to (C_3);
			\draw[->, thick] (nx_j)   to (C_3);
			\draw[->, thick] (x_k)  to (C_3);

			\draw[->, thick] (nx_i)   to (C_4);
			\draw[->, thick] (nx_j)   to (C_4);
			\draw[->, thick] (x_k)  to (C_4);

			\draw[->, thick] (x_i)   to (C_5);
			\draw[->, thick] (x_j)   to (C_5);
			\draw[->, thick] (x_k)  to (C_5);
		}
	\end{tikzpicture}
\end{figure}

\end{frame}

\begin{frame}{Problém}
    \begin{figure}
\begin{tikzpicture}[
	stff/.style={circle, draw=black, thick, minimum size=30},
	stffc/.style={rectangle, draw=black, thick, minimum size=30},
 scale=1.3,transform shape
	]
		\node	(x)      {};
		\node	(y)    [below=of x]  {};
		\node	(z)    [below=of y]  {};

		%Nodes
		\node[stffc]	(C_1)    [right=of y,yshift=-5,xshift=100]  {$ C $};
		\node	(a)      [right=of C_1,xshift=100] {};
		
		\alert<3->{\draw[->, thick] (x)  to (C_1);
		\alert<2->{\draw[->, thick] (y)  to (C_1);}
		\draw[->, thick] (z)  to (C_1);}
	\end{tikzpicture}
\end{figure}

\end{frame}

\begin{frame}{Řešení}
    \begin{figure}
\begin{tikzpicture}[
	stff/.style={circle, draw=black, thick},
	stffc/.style={rectangle, draw=black, thick, minimum size=30},
 scale=1.3,transform shape
	]
		\node	(x)      {};
		\node	(y)    [below=of x]  {};
		\node	(z)    [below=of y]  {};

		%Nodes
		\node[stff]	(C2)    [right=of y,xshift=90]  {};
		\node[stff]	(C1)    [above=of C2,xshift=30,yshift=-20]  {};
		\node[stff]	(C3)    [below=of C2,xshift=30, yshift=20]  {};
		\node	(a)      [right=of C1,xshift=100] {};
		
		\draw[->, thick] (x)  to (C1);
		\draw[->, thick] (y)  to (C2);
		\draw[->, thick] (z)  to (C3);

		\draw[<->, thick] (C1)  to (C3);
		\draw[<->, thick] (C2)  to (C3);
		\draw[<->, thick] (C1)  to (C2);


		\onslide<2->{
			\node[stff]	(D1)    [right=of C1,yshift=-12]  {};
			\node[stff]	(D2)    [right=of C3, yshift=12]  {};

			\draw[->, thick] (D1)	to (C1);
			\draw[->, thick] (D2)	to (C1);
			\draw[->, thick] (D1)	to (C2);
			\draw[->, thick] (D2)	to (C2);
			\draw[->, thick] (D1)	to (C3);
			\draw[->, thick] (D2)	to (C3);
		}

	\end{tikzpicture}
\end{figure}

\end{frame}

\section*{Děkuji za pozornost}

\end{document}
