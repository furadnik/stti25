\documentclass[]{beamer}

% setup basic metainfo
\title{Cooperative Game Theory \\ \& Optimism Bias}
\author{Filip Úradník}
\date{\today}

\input{template}
% \input{color}
% \documentclass [14pt,xcolor=dvipsnames,aspectratio=169]{beamer} 
\usetheme{metropolis}
\setbeamertemplate{caption}{\raggedright\insertcaption\par}
\setbeamertemplate{frame footer}{\color{lightgray}\texttt{uradnik@kam.mff.cuni.cz}}
\metroset{block=fill}

\definecolor{mDarkBrown}{RGB}{45, 8, 36}
\definecolor{mDarkTeal}{RGB}{45, 8, 36}
\definecolor{mLightBrown}{RGB}{229, 0, 168}
\definecolor{mLightGreen}{RGB}{229, 0, 168}

% english localisation
\usepackage[english]{babel}
% \renewcommand{\lstlistingname}{Code sample}
% \renewcommand{\lstlistlistingname}{List of Code Samples}
% also: defi, thm, algo, glossary

\newtheorem{defi}[equation]{Definition}
\newtheorem{notation}[equation]{Notation}

% \newtheorem{lemma}[equation]{Lemma}
\newtheorem{algo}[equation]{Algorithm}
\newtheorem{obs}[equation]{Observation}
\newtheorem{prop}[equation]{Proposition}
\newtheorem{invar}[equation]{Invariant}
\newtheorem{cor}[equation]{Corollary}

\newtheorem{examp}[equation]{Example}
% \newtheorem{fact}[equation]{Fact}
% \newtheorem{note}[equation]{Note}


\newcommand{\glostitle}{Glossary}


\usepackage[natbib=true,style=authoryear,backend=bibtex,useprefix=true]{biblatex}
\addbibresource{bibliography.bib}
\setbeamertemplate{caption}{\raggedright\insertcaption\par}

\def\blue#1{{\color{blue} #1}}
\def\orange#1{{\color{orange} #1}}
\def\red#1{{\color{red} #1}}
\def\s{\mathcal{S}}

\begin{document}
\maketitle

\begin{frame}{Cooperative Game Theory -- Motivation}
	\begin{itemize}
		\item There is a group of players with common goal.
		\item Players can form \emph{coalitions}.
		\item A formed coalition receives \emph{reward}.
	\end{itemize}
	\vspace{2em}
	\begin{definition}[Cooperative Game]
		A \emph{cooperative game} is a tuple $ \left( N,v \right) $, where \begin{enumerate}
			\item $ N = \left\{ 1, \ldots, n \right\} $ is the set of \emph{players},
			\item $ v: \pot N \to \R $ is the \emph{characteristic function}, with $ \fce{v}{\emptyset} = 0 $.
		\end{enumerate}
	\end{definition}
\end{frame}

\begin{frame}{Cooperative Game Theory}
	In CGT, we try to answer the following:
	\begin{itemize}[ ]
		\item Which coalition should form?
		\item How should the received reward be distributed?
	\end{itemize}
	Battle plan:
	\begin{enumerate}
		\item We assume the grand coalition $ N $ forms.
		\item We suggest a distribution of $ \fce{v}{N} $ to the players.
		\item We analyze whether this coalition is stable.
	\end{enumerate}
\end{frame}

\begin{frame}{Payoff Vectors}
	Distribution of $ \fce{v}{N} $ is formalised as a \emph{payoff vector}
	\[
		x \in \R^n.
	\]
	We require \emph{efficiency}, meaning \[
		\sum_{i \in N}^{} x_i = \fce{v}{N}.
	\]
	We study \emph{solution concepts}---payoff vectors with given properties \[
		f: \Gamma^n \to {\R^n} \qquad \text{or} \qquad f: \Gamma^n \to \pot {\R^n}.
	\]
\end{frame}

\begin{frame}{Stability}
	\begin{definition}[Core]
		The \emph{core} is a solution concept $ \mathcal{C}: \Gamma^n \to \pot {\R^n} $ such that, for an efficient $ x \in \R^n $, \[
			x \in \mathcal{C}(N,v) \qquad \iff \qquad \left( \forall S \subseteq N \right)\quad \sum_{i \in S}^{} x_i \geq v(S).
		\]
	\end{definition}
\end{frame}

\begin{frame}{Fairness}
	A fair solution concept should satisfy the following axioms:
	\begin{enumerate}
		\item efficiency, i.e., $\sum_{i \in N}^{} \shapley[i]{v} = \fce{v}{N},$
		\item symmetry, i.e., $  \forall i,j \in N, \forall S \subseteq N \setminus \left\{ i,j \right\} $\[
				 \fce{v}{S \cup i} = \fce{v}{S \cup j} \quad\implies\quad \shapley[i]{v} = \shapley[j]{v} ,
			\]
		\item null player, i.e., $ \forall i \in N, \forall S \subseteq N \setminus \left\{ i \right\} $
			\[
				\fce{v}{S} = \fce{v}{S \cup i} \implies \shapley[i]{v} = 0,
			\]
		\item additivity, i.e., $ \forall v,w \in \Gamma^ n, $
			\[
				\shapley{v + w} = \shapley{v} + \shapley{w}.
			\]
	\end{enumerate}
\end{frame}

\begin{frame}{Fairness -- Shapley Value}
	\begin{definition}[Shapley Value]
		The \emph{Shapley value} is a solution concept $ \Shapley: \Gamma^n \to \R^n $, where \[
			\shapley[i]v \deq \sum_{S \subseteq N \setminus i}^{} \frac{\absolute S! \left( n-\absolute S-1 \right)!}{n!} \left( \fce{v}{S} - \fce{v}{S \cup i} \right).
		\]
	\end{definition}

	\begin{theorem}[Shapley]
		The Shapley value $ \Shapley $ is the unique function satisfying all the axioms of fairness stated before.
	\end{theorem}
\end{frame}

\begin{frame}{Further Questions to Ask}
	\begin{itemize}[ ]
		\item When is the core non-empty?
		\item When is the Shapley value in the core?
	\end{itemize}
	\vspace{1em}
	\begin{theorem}
		If the characteristic function is \emph{supermodular}, i.e., \[
			\left( \forall S,T \subseteq N \right)\quad \fce{v}{S} + \fce{v}{T} \leq \fce{v}{S \cup T} + \fce{v}{S \cap T},
		\]
		then $ \mathcal{C}(v) $ is non-empty, and specifically, \[
			\shapley v \in \mathcal{C}(v).
		\]
	\end{theorem}
\end{frame}

\section{Optimism Bias}

\begin{frame}{What Is Wrong?}
	\begin{definition}[Cooperative Game]
		A \emph{cooperative game} is a tuple $ \left( N,v \right) $, where \begin{enumerate}
			\item $ N = \left\{ 1, \ldots, n \right\} $ is the set of \emph{players},
			\item $ v: \pot N \to \R $ is the \emph{characteristic function}, with $ \fce{v}{\emptyset} = 0 $.
		\end{enumerate}
	\end{definition}
\end{frame}

\begin{frame}{Incomplete Information}
	Assume we only know the values of some coalitions.
	\vspace{3em}
	\begin{definition}[Incomplete Cooperative Game]
		An \emph{incomplete cooperative game} is a tuple $ \left( N,\k,v \right) $, where \begin{enumerate}
			\item $ N = \left\{ 1, \ldots, n \right\} $ is the set of \emph{players},
			\item $ v: \pot N \to \R $ is the \emph{characteristic function}, with $ \fce{v}{\emptyset} = 0 $,
			\item $ \k \subseteq \pot N $ are the \emph{known values}.
		\end{enumerate}
	\end{definition}
\end{frame}

\begin{frame}{Some Necessary Constraints}
	We assume \emph{superadditivity} of $ v $: \[
		\left( \forall S,T \subseteq N, S \cap T = \emptyset \right)\qquad \fce{v}{S} + \fce{v}{T} \leq \fce{v}{S \cup T}.
	\]
	We further assume minimal information to be present: \[
		\k \supseteq \k_0,
	\]
	$ \k_0 = \left\{ \emptyset, N \right\} \cup \left\{ \left\{ i \right\} \suchthat i \in N \right\} $.
\end{frame}

\begin{frame}{Superadditive Extensions -- Candidates for Real Values}
		\begin{center}
		\begin{tikzpicture}[] %%[scale=4] ONLY changes distances, not the canvas
    % Define the coordinates of the vertices
    \coordinate (A) at (0, 0);
    \coordinate (B) at (4.3, 0);
    \coordinate (C) at (7, 3.5);
    \coordinate (D) at (5, 6);
    \coordinate (E) at (2, 5);
    \coordinate (F) at (0, 2.5);
    \coordinate (v) at (3, 3.5);
    \coordinate (v1) at (B);
    \coordinate (v2) at (D);
    \coordinate (v3) at (E);
    
    % Draw and fill the hexagon
		\filldraw[very thick, fill=white!85!mDarkBrown, draw=mDarkBrown] (A) -- (B) -- (C) node[anchor=south,yshift=30]{$\mathbb{S}^n(v,\k)$}-- (D) -- (E) -- (F) -- cycle;

    % \filldraw[mDarkBrown] (A) circle (1pt) node[anchor=north]{$A$};
    % \filldraw[mDarkBrown] (B) circle (1pt) node[anchor=north]{$B$};
    % \filldraw[mDarkBrown] (C) circle (1pt) node[anchor=west]{$C$};
    % \filldraw[mDarkBrown] (D) circle (1pt) node[anchor=south]{$D$};
    % \filldraw[mDarkBrown] (E) circle (1pt) node[anchor=south]{$E$};
    % \filldraw[mDarkBrown] (F) circle (1pt) node[anchor=east]{$F$};

		\filldraw[mDarkBrown] (v) circle (1.5pt) node[anchor=south]{${v}$};
		\filldraw[red] (v1) circle (1.5pt) node[anchor=north]{${v_1}$};
		\filldraw[orange] (v2) circle (1.5pt) node[anchor=south]{${v_2}$};
		\filldraw[blue] (v3) circle (1.5pt) node[anchor=east]{${v_3}$};
		\end{tikzpicture}
	\end{center}
\end{frame}


\begin{frame}{Utopian Gap}
    \begin{itemize}[ ]
        \item Players are \emph{optimistically biased}.
        \item As a group made $ v(N) $.
				\item Players demand $ \red{\shapley[1]{v_1}}, \orange{\shapley[2]{v_2}}, \blue{\shapley[3]{v_3}}, \ldots $
    \end{itemize}
		\vspace{2em}

		\begin{definition}[Utopian Gap]
        The \emph{utopian gap} of $ \left( N, \k, v \right) $ is \[
					\mathcal{G}(v, \k) := \sum_{i \in N} \shapley[i]{v_i} - v(N),
        \]
				where $v_i \deq \argmax_{v \in \mathbb{S}(v,\k)} \shapley[i]v$.
			\end{definition}
\end{frame}

\begin{frame}{Reducing $ \mathcal{G} $ -- Setting}
	\begin{itemize}[ ]
		\item We have $ v \sim \mathcal{F} $, where $ \mathcal{F} $ is a distribution of superadditive games.
		\item We only know $ \k \supseteq \k_0 $ values of it.
		\item We have a budget $ \tau $ of how many values we can learn.
	\end{itemize}
\end{frame}

\begin{frame}{Reducing Optimism Bias -- Offline Approach}
	In the simplest case, we can minimize the expected value: \[
		\s^* = \argmin_{\s, \absolute{\s} = \tau} \E_{v \sim \mathcal{F}} \left[ \mathcal{G}(v,\k \cup \s) \right].
	\]

	\vspace{2em}
	We call this the \emph{Offline} approach.
\end{frame}

\begin{frame}{Reducing Optimism Bias -- Online Approach}
	It is \textquote{inefficient} to learn all values at once.

	The \emph{Online} approach seeks to find a policy $ \pi $ which selects the next value to learn based on the values already known.

	A solution to the Online approach can be approximated using reinforcement learning (we use the PPO algorithm).
\end{frame}

\begin{frame}{Reducing Optimism Bias -- Results for $ \mathcal{F} = \texttt{factory} (5) $}
	\begin{center}
		\includegraphics[width=\textwidth]{figures/factory5.pdf}
	\end{center}
\end{frame}

\begin{frame}{Reducing Optimism Bias -- Results for $ \mathcal{F} = \texttt{supermodular} (5) $}
	\begin{center}
		\includegraphics[width=\textwidth]{figures/convex5.pdf}
	\end{center}
\end{frame}

\begin{frame}{Reducing Optimism Bias -- \textsc{Largest Coalitions} Heuristic}
	\begin{center}
		\includegraphics[width=\textwidth]{figures/convex_linear.pdf}
	\end{center}
\end{frame}

\section{Conclusion}

\begin{frame}[allowframebreaks]{References}
    \nocite{*}
    \printbibliography[heading=none]
\end{frame}

\begin{frame}{Acknowledgments}
	This work was supported by CoSP, a project funded by European Union’s Horizon
	2020 research and innovation programme, grant agreement No. 823748.

	Thank you to DIMACS for organizing the REU 2024 program.

	Thank you as well to David Sychrovský, Martin Černý, and Jakub Černý, who
	were my co-authors on the paper which inspired this talk \citep{uradnik2024reducing}.
\end{frame}

\section{Thank You!}

\end{document}
